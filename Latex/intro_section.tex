% Credit Risk Analysis and Prediction Framework
% Introductory Section
% UtkarshDubeyGIT et al.

\section{Introduction}

Credit risk analysis is a cornerstone of modern financial decision-making, directly impacting the stability and profitability of lending institutions. The ability to accurately predict the likelihood of loan default is essential for banks, credit card companies, and fintech organizations. In recent years, the integration of machine learning (ML) and fuzzy logic has enabled the development of more robust, interpretable, and human-centric credit risk models.

This research presents an explainable fuzzy credit-risk prediction framework (X-FuzzyScore) that combines fuzzy reasoning, ML ensemble methods, and explainability techniques (e.g., SHAP) to deliver transparent, actionable insights for credit scoring. The system is designed to:
\begin{itemize}
    \item Predict credit-risk and loan-default probability for individuals and companies.
    \item Provide interpretable predictions in human language and visual formats.
    \item Integrate fuzzy reasoning (e.g., ``high income'', ``medium debt'') with ML accuracy.
    \item Offer a dashboard for model results, fuzzy rules, and SHAP explanations.
\end{itemize}

\subsection{Context and Motivation}

As outlined in the team briefing, the project aims to address the lack of transparency in traditional credit scoring models by introducing explainable AI (XAI) techniques. The framework leverages multiple datasets (German Credit, Taiwan Credit Card Default, LendingClub Loan Data) to ensure both interpretability and scalability. Data preprocessing includes feature alignment, normalization, categorical encoding, and dataset integration.

\subsection{System Architecture}

\begin{figure}[h]
    \centering
    \includegraphics[width=0.9\textwidth]{system_architecture.png}
    \caption{High-level architecture of the X-FuzzyScore credit risk prediction system.}
\end{figure}

The architecture consists of:
\begin{enumerate}
    \item Data Preprocessing: Feature alignment, normalization, encoding, and integration.
    \item Fuzzy Layer: Linguistic variable definition and fuzzy rule application.
    \item ML Ensemble Layer: XGBoost, LightGBM, and Random Forest for prediction.
    \item Explainability Layer: SHAP/LIME for feature attribution.
    \item Visualization Frontend: Dashboard with risk gauge, SHAP bar plots, and fuzzy rule viewer.
\end{enumerate}

\subsection{Dataset Overview}

\begin{table}[h]
    \centering
    \begin{tabular}{|l|l|l|l|}
        \hline
        \textbf{Dataset} & \textbf{Source} & \textbf{Size} & \textbf{Use} \\
        \hline
        German Credit & UCI ML Repository & $\sim$1,000 & Small interpretable dataset \\
        Taiwan Credit Card Default & UCI ML Repository & 30,000 & Large-scale testing \\
        LendingClub Loan Data & Kaggle & 100,000+ & Real-world validation \\
        \hline
    \end{tabular}
    \caption{Datasets used for credit risk analysis.}
\end{table}

\subsection{Feature Description}

The Taiwan Credit Card Default dataset includes features such as credit limit, sex, education, marriage, age, payment history, bill amounts, and payment amounts. Categorical variables are encoded and outliers are handled during preprocessing. For hybrid models, a fuzzy risk score is added as an additional feature.

\subsection{Mathematical Formulation}

Let $X$ be the feature matrix, $y$ the target (default status), and $f_{ML}$ the machine learning model. The fuzzy risk score $f_{fuzzy}(X)$ is computed using fuzzy rules:
\begin{equation}
    \text{Risk} = f_{ML}(X) + \lambda \cdot f_{fuzzy}(X)
\end{equation}
where $\lambda$ controls the influence of fuzzy logic.

\subsection{Performance Comparison}

\begin{table}[h]
    \centering
    \begin{tabular}{|l|l|l|l|l|l|}
        \hline
        \textbf{Model} & \textbf{Type} & \textbf{Accuracy} & \textbf{Precision} & \textbf{Recall} & \textbf{ROC-AUC} \\
        \hline
        LightGBM + Fuzzy & Hybrid & 0.76 & 0.46 & 0.61 & 0.77 \\
        XGBoost + Fuzzy & Hybrid & 0.76 & 0.46 & 0.61 & 0.77 \\
        RF + Fuzzy & Hybrid & 0.78 & 0.51 & 0.56 & 0.77 \\
        Random Forest & Baseline & 0.78 & 0.50 & 0.55 & 0.77 \\
        Logistic Regression & Baseline & 0.77 & 0.49 & 0.55 & 0.75 \\
        \hline
    \end{tabular}
    \caption{Model performance comparison on Taiwan Credit Card Default dataset.}
\end{table}

\subsection{Visualization Examples}

\begin{figure}[h]
    \centering
    \includegraphics[width=0.45\textwidth]{default_distribution.png}
    \includegraphics[width=0.45\textwidth]{education_distribution.png}
    \caption{Distribution of default payments and education levels.}
\end{figure}

\begin{figure}[h]
    \centering
    \includegraphics[width=0.45\textwidth]{credit_limit_histogram.png}
    \includegraphics[width=0.45\textwidth]{credit_limit_boxplot.png}
    \caption{Credit limit distribution and box plot.}
\end{figure}

\begin{figure}[h]
    \centering
    \includegraphics[width=0.6\textwidth]{correlation_heatmap.png}
    \caption{Correlation heatmap of dataset features.}
\end{figure}

\subsection{Expected Outputs}

\begin{itemize}
    \item Probability: $0.87 \rightarrow 87\%$ chance of repayment
    \item Risk Label: ``Low Risk'', ``Medium Risk'', ``High Risk''
    \item Fuzzy Rules Triggered: ``IF income = high AND debt = low $\rightarrow$ risk = low (activation 0.82)''
    \item SHAP Explanation: income $-0.18$ $\rightarrow$ reduced risk; debt $+0.07$ $\rightarrow$ increased risk
    \item Visualization: Dashboard with gauge, SHAP bars, fuzzy memberships
\end{itemize}

\section{Conclusion}

This introductory section establishes the context, motivation, and technical foundation for the X-FuzzyScore credit risk analysis framework. The following sections will detail methodology, experiments, and results.
